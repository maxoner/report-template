\section{Расчет биологической защиты} %TODO: убрать для репорта
\subsection{Постановка задачи}
Необходимо рассчитать дозу облучения при стационарном режиме работы ЯЭУ ВВЭР-1000 за биологической защитой

\subsection{Построение расчетной модели биологической защиты}
Для формирования расчетной модели рассмотрим компоновку элементов и помещений ЯЭУ с РУ ВВЭР-1000.

\begin{figure}[H]
	\begin{center}
		\includegraphics{razrez_bio.png}
		\caption{
			Общая компоновка энергоблока с РУ ВВЭР-1000 (Южно-Украинская АЭС): \\
			1 — реактор; 2 — машина для перегрузки топлива; 3 — подъемный кран реакторного отделения; 4 — компенсатор давления, 5 — барботер; 6 — деаэратор; 7 — гидроемкость, 8 — турбогенератор; 9 — подъемный кран машинного зала; 10 — регенеративные подогреватели, 11 —  защитная оболочка
		}
		\label{pic:razrez_bio} % название для ссылок внутри кода
	\end{center}
\end{figure}
% Монахов Атомные электрические станции и их технологическое оборудование

\paragraph{Элементы компоновки вокруг реактора}

Рассмотрим основные элементы защиты, внешние по отношению к ВВЭР-1000 в сборе. Корпус реактора установливается в \textit{бетонную шахту} (рис \ref{pic:beton_shakhta}), которая играет роль основной опоры и крепления реактора с учетом сейсмических нагрузкок, а также биологической защиты от излучения со стороны АЗ. Между корпусом реактора и шахтой имеется кольцевой зазор, предназначенный для периодического контроля металла корпуса в связи с требованиями правил. Шахта резделена по высоте на два объема разделительным сильфоном: 
\begin{itemize}
	\item Верхний, снабжен гидрозатвором и соединяется с бассейном выдержки. При перегрузке верхний объем шахты вместе с бассейном заливается водой.
	\item Нижний, условно разделяемый фермой опорной на шахту зоны патрубков и шахту цилиндрической части корпуса. Соединяется проемом, снабженным герметичной дверью, с помещением для машины осмотра корпуса.
\end{itemize}
В помещении зоны патрубков
биологическая защита выполнена из металлических коробов, заполненных специальным составом, в который входят серпентинитовая галя, кристаллический карбид бора, дробь чугунная литая. В районе активной зоны применяется «сухая» защита, которая представляет из себя слой серпентинитового бетона толщиной
720 мм и высотой 4,7 м, облицованного металлической оболочкой. Такой бетон обладает высокой радиационной стойкостью, что позволяет удовлетворить требования по нейтронной защите. \cite{лескин2011физические}

\begin{figure}[H]
	\begin{center}
		\includegraphics[scale=0.4]{beton.jpg}
		\caption{
			Бетонная шахта реактора 
		}
		\label{pic:beton_shakhta} % название для ссылок внутри кода
	\end{center}
\end{figure}

Все оборудование первого контура заключено в цилиндрическую оболочку, в верхней части которой расположен грузоподъемный поворотный кран. Между реакторным и машинным залами располагается этажерка электротехнических устройств, где размещены также деаэраторы и различные лаборатории.

\paragraph{Корпус и внутрикорпусные элементы компоновки} Корпус представляет собой вертикальный герметичный сосуд
цилиндрической формы с эллиптическими днищем и крышкой  с наружним диаметром 4535 мм, высотой 10.897 м и толщиной 192 мм в цилиндрической части и 210 мм в районе патрубков \cite{лескин2011физические}. В качестве основного материала используется сталь сталь 15Х2НМФА. е. Вся внутренняя поверхность корпуса покрыта
антикоррозионной наплавкой из нержавеющей стали толщиной не
менее 8 мм. В местах соприкосновения корпуса с крышкой, шахтой, уплотнительными прокладками, в местах приварки
кронштейнов, деталей крепления трубок КИП, на поверхности разделительного кольца выполнена наплавка толщиной не менее
15 мм. Внутрь реактора также устанавливается шахта, которая  представляет собой цилиндрическую обечайку с фланцем и эллиптическим днищем, в котором закреплены 163 опорные
трубы (стаканы) с шагом 236 мм, верхние части которых образуют
опорную плиту для установки и дистанционирования кассет активной зоны. Материал шахты – сталь 08Х18Н10Т толщиной 55 мм.

\paragraph{Построение одномерной модели} Для построения расчетной модели был определен ряд значимых элементов конструкции реакторной установки с точки зрения нейтронной защиты. 
Основная доля нейтронного излучения в реакторе приходится на нейтроны теплового спектра. Для таких энергий хрошими поглотителями являются кадмий, графит, бетон. 
Присутствующее гамма-излучение для своего эффективного поглощения требует свинец и подобные высокоплотные материалы. Таким образом были выбран слои биологической защиты, представленные в таблице \ref{tabular:bio-sec-layers}:

\begin{table}[H]
	\caption{Слои биологической защиты}
	\begin{center}
        \begin{tabular}{|l|c|c|c|}
        \toprule
         Название & Материал & Размер & Плотность, $\text{г}/\text{см}^3$ \\ 
         \midrule
         \hline
         Оболочка ТВЭЛ & Zr1Nb & 0.65 & 6,55\\ 
         \hline
         Внутрикорпусная шахта & сталь 08Х18Н10Т & 55 мм & 7.9 \\ 
         \hline
         Корпус & сталь 15Х2НМФА & 200 мм & 7.8 \\ 
         \hline
         Бетонная шахта & серпентинитовый бетон & 720 мм & 2.35 \\ 
         %\hline
         %Герметичная оболочка & бетон &  & 2.5 \\ 
         \bottomrule
		\end{tabular}
		\label{tabular:bio-sec-layers}
	\end{center}
\end{table}

\subsection{Расчет дозы нейтронов из активной зоны реактора}

\begin{table}[H]
	\caption{Основные нейтронно-физические параметры}
	\begin{center}
        \begin{tabular}{|l|c|}
        \toprule
         Параметр & Значение \\ 
         \midrule
         \hline
         Тепловая мощность реактора, МВт & $2.904 \cdot 10^3$ \\ 
         \hline
         Средняя энергия, выделяющаяся в одной реакции деления, МэВ &  \\ 
         \hline
         Среднее число нейтронов деления на середину кампани & \\
         \hline
         Коэффициент размножения &  \\ 
         \hline
         Доля нейтронов спектра деления в спектре утечки &  \\ 
         \hline
         Среднее число гамма-квантов деления на середину кампании &  \\ 
         \bottomrule
		\end{tabular}
		\label{tabular:turbine}
	\end{center}
\end{table}