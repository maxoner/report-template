\section*{ВВЕДЕНИЕ}
\addcontentsline{toc}{section}{\protect ВВЕДЕНИЕ}
В настоящее время в атомной промышленности большое распространение имеют водо-водяные корпусные реакторы, как кипящие (американские PWR), так и некипящие (русские ВВЭР). Благодаря использованию воды в качестве теплоносителя и замедлителя такие установки относительно дешевые в эксплуатации и достаточно безопасные ввиду возможности обеспечения отрицательного коэффициента реактивности. Отдельно стоит отметить установку ВВЭР-1000, имеющую наибольшее распространение среди реакторов такой серии (37 действующих установок из 60-и ВВЭР). Проект хорошо себя зарекомендовал и находится в эксплуатации с 1980 года, когда был установлен первый реактор такого типа на пятом энергоблоке Нововоронежской АЭС, новые установки данной модели производятся по сей день. При строительстве первых реакторов ВВЭР-1000 проектом закладывлся срок эксплуатации в 30 лет, сейчас же учитывая современные научные знания срок службы таких установок и идентичных новых был увеличен до 60 лет. При столь длинных сроках эксплуатации, на которые рассчитаны РУ ВВЭР-1000 возникает потребность адаптировать существующие к современным энергетическим потребностям. 

Проектом такие РУ изначально рассчитывались на большую мощность, что позволяет по необходимости эффективнее утилизировать мощностные ресурсы установки, благодаря чему можно сэкономить на строительстве новых энергоблоков. Так, например, на балаковской АЭС для ВВЭР-1000 серийного проекта В-320 успешно было проведено мероприятие по повышению мощности на 4\% \cite{104балаково}. 

Таким образом проблема вертикального масштабирования по мощности наиболее актуальна для реакторов типа ВВЭР-1000, что определяет и \textit{актуальность} данной работы. 

\textit{Целью} данной работы являлось расчет и исследования температурных полей реакторной установки ВВЭР-1000 при работе на повышенной мощности, а также на пониженной при отключении одного из четырех и двух из четырех ГЦН. \textit{Практическая значимость} данной работы состоит в определении максимальных температур и выводов о работоспособности РУ в исследуемых режимах работы. На основе поставленной цели были сформированы следующие задачи:
\begin{itemize}
    \item Произвести теплофизический и нейтронно-физический расчет ВВЭР-1000 и определить характеристики РУ в номинальном режиме работы 
    \item Ознакомится с расчетной моделью программного комплекса ТРЕТОН и прозвести валидационный расчет РУ в номинальном режиме.
    \item Произвести рассчетное исследование работы реактора в режиме повышенной мощности, определить максимальные температуры и их превышения и сделать вывод о возможности работы реактора в таком режиме
    \item Произвести рассчетное исследование работы реактора при отключении одного из трех и двух из трех, произвести анализ температурных полей и сделать вывод о возможности работы РУ при пониженной мощности
\end{itemize}