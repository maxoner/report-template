\section*{РЕФЕРАТ}
% \addcontentsline{toc}{section}{\protect РЕФЕРАТ}%
Выпускная квалификационная работа содержит 61 - страниц, 55 - рисунков, 11 - таблиц и 9 -источников.

РЕАКТОР ВВЭР-1000, АКТИВНАЯ ЗОНА, ГИДРОДИНАМИКА, ТЕПЛООБМЕН, ОТКЛЮЧЕНИЕ ГЦН, ПОВЫШЕНИЕ МОЩНОСТИ,  МОДЕЛИРОВАНИЕ, ПРОГРАММНЫЙ МОДУЛЬ ТРЕТОН

В работе выполнено моделирование теплогидравлических параметров активной зоны ВВЭР-1000 в номинальном режиме, а так же в не номинальных режимах с увеличением мощности реактора, и с отключением одного из четырех и двух их четырех ГЦН.
\begin{enumerate}
    \item В первой главе приведен обзор конструкции и предоставлена общая информация реакторной установки ВВЭР-1000
    \item Во второй главе осуществлен теплофизический расчет реактора ВВЭР-1000 в номинальном режиме
    \item В третий главе осуществлен теплофизический расчет ВВЭР-1000 на программном модуле ТРЕТОН в номинальном режиме, а так же сравнения данных с аналитическим расчетом, осуществлен расчет в неноминальных режимах с увеличением мощности на 105\%, 110\%, 115\% и отключением одного и двух стоящих друг на против друга ГЦН. 
\end{enumerate}
