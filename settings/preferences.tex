% настройки polyglossia
\setdefaultlanguage{russian}
\setotherlanguage{english}

% локализация
\addto\captionsrussian{
	\renewcommand{\figurename}{Рисунок}%
	\renewcommand{\partname}{Глава}
	\renewcommand{\contentsname}{\centerline{Содержание}}
	\renewcommand{\listingscaption}{Листинг}
}

% основной шрифт документа
\setmainfont{Liberation Serif}
\newfontfamily\cyrillicfont{Liberation Serif}[Script=Cyrillic]

% перечень использованных источников
\addbibresource{refs.bib}

% настройка полей
\geometry{top=2cm}
\geometry{bottom=2cm}
\geometry{left=2cm}
\geometry{right=2cm}
\geometry{bindingoffset=0cm}

% настройка ссылок и метаданных документа
\hypersetup{unicode=true,colorlinks=true,linkcolor=blue,citecolor=green,filecolor=magenta,urlcolor=cyan,
	pdftitle={\docname},
	pdfauthor={\studentname},
	pdfsubject={\subject},
	pdfcreator={\studentname},
	pdfproducer={Overleaf},
	pdfkeywords={\subject}
}

% настройка подсветки кода и окружения для листингов
\usemintedstyle{colorful}
\newenvironment{code}{\captionsetup{type=listing}}{}

% шрифт для листингов с лигатурами
\setmonofont{FiraCode-Regular.otf}[
	SizeFeatures={Size=10},
	Path = templates/,
	Contextuals=Alternate
]

% оформления подписи рисунка
\captionsetup[figure]{labelsep = period}

% подпись таблицы
% \DeclareCaptionFormat{hfillstart}{\hfill#1#2#3\par}
% \captionsetup[table]{format=hfillstart,labelsep=newline,justification=centering,skip=-10pt,textfont=bf}

% путь к каталогу с рисунками
\graphicspath{{fig/}}

% Внесение titlepage в учёт счётчика страниц
\makeatletter
\renewenvironment{titlepage} {
	\thispagestyle{empty}
}
\makeatother

\counterwithin{figure}{section}
\counterwithin{table}{section}

\titlelabel{\thetitle.\quad}

% для удобного конспектирования математики
\mathtoolsset{showonlyrefs=false}
\theoremstyle{plain}
\newtheorem{theorem}{Теорема}[section]
\newtheorem{proposition}[theorem]{Утверждение}
\theoremstyle{definition}
\newtheorem{corollary}{Следствие}[theorem]
\newtheorem{problem}{Задача}[section]
\theoremstyle{remark}
\newtheorem*{nonum}{Решение}

% настоящее матожидание
\newcommand{\MExpect}{\mathsf{M}}

% объявили оператор!
\DeclareMathOperator{\sgn}{\mathop{sgn}}

% перенос знаков в формулах (по Львовскому)
\newcommand*{\hm}[1]{#1\nobreak\discretionary{} {\hbox{$\mathsurround=0pt #1$}}{}}


% \numberwithin{equation}{section}equation

\newlength{\myl}
\let\origequation=\equation
\let\origendequation=\endequation

\DeclareSymbolFont{lettersA}{U}{txmia}{m}{it} 
\SetSymbolFont{lettersA}{bold}{U}{txmia}{bx}{it} 
\DeclareFontSubstitution{U}{txmia}{m}{it} 
\DeclareSymbolFontAlphabet{\mathfrak}{lettersA}

\DeclareMathSymbol{\alphaup}{\mathord}{lettersA}{11}
\DeclareMathSymbol{\betaup}{\mathord}{lettersA}{12}
\DeclareMathSymbol{\gammaup}{\mathord}{lettersA}{13}
\DeclareMathSymbol{\deltaup}{\mathord}{lettersA}{14}
\DeclareMathSymbol{\epsilonup}{\mathord}{lettersA}{15}
\DeclareMathSymbol{\zetaup}{\mathord}{lettersA}{16}
\DeclareMathSymbol{\etaup}{\mathord}{lettersA}{17}
\DeclareMathSymbol{\thetaup}{\mathord}{lettersA}{18}
\DeclareMathSymbol{\iotaup}{\mathord}{lettersA}{19}
\DeclareMathSymbol{\kappaup}{\mathord}{lettersA}{20}
\DeclareMathSymbol{\lambdaup}{\mathord}{lettersA}{21}
\DeclareMathSymbol{\muup}{\mathord}{lettersA}{22}
\DeclareMathSymbol{\nuup}{\mathord}{lettersA}{23}
\DeclareMathSymbol{\xiup}{\mathord}{lettersA}{24}
\DeclareMathSymbol{\piup}{\mathord}{lettersA}{25}
\DeclareMathSymbol{\rhoup}{\mathord}{lettersA}{26}
\DeclareMathSymbol{\sigmaup}{\mathord}{lettersA}{27}
\DeclareMathSymbol{\tauup}{\mathord}{lettersA}{28}
\DeclareMathSymbol{\upsilonup}{\mathord}{lettersA}{29}
\DeclareMathSymbol{\phiup}{\mathord}{lettersA}{30}
\DeclareMathSymbol{\chiup}{\mathord}{lettersA}{31}
\DeclareMathSymbol{\psiup}{\mathord}{lettersA}{32}
\DeclareMathSymbol{\omegaup}{\mathord}{lettersA}{33}
\DeclareMathSymbol{\varepsilonup}{\mathord}{lettersA}{34}
\DeclareMathSymbol{\varthetaup}{\mathord}{lettersA}{35}
\DeclareMathSymbol{\varpiup}{\mathord}{lettersA}{36}
\DeclareMathSymbol{\varrhoup}{\mathord}{lettersA}{37}
\DeclareMathSymbol{\varsigmaup}{\mathord}{lettersA}{38}
\DeclareMathSymbol{\varphiup}{\mathord}{lettersA}{39}

\renewcommand{\alpha}{\alphaup}
\renewcommand{\beta}{\betaup}
\renewcommand{\gamma}{\gammaup}
\renewcommand{\delta}{\deltaup}
\renewcommand{\epsilon}{\varepsilonup}
\renewcommand{\zeta}{\zetaup}
\renewcommand{\eta}{\etaup}
\renewcommand{\theta}{\theteaup}
\renewcommand{\iota}{\iotaup}
\renewcommand{\kappa}{\kappaup}
\renewcommand{\lambda}{\lambdaup}
\renewcommand{\mu}{\muup}
\renewcommand{\nu}{\nuup}
\renewcommand{\xi}{\xiup}
\renewcommand{\pi}{\piup}
\renewcommand{\rho}{\rhoup}
\renewcommand{\sigma}{\sigmaup}
\renewcommand{\tau}{\tauup}
\renewcommand{\upsilon}{\upsilonup}
\renewcommand{\phi}{\varphiup}
\renewcommand{\chi}{\chiup}
\renewcommand{\omega}{\omegaup}
\renewcommand{\vartheta}{\varthetaup}
\renewcommand{\varpi}{\varpiup}
\renewcommand{\varrho}{\varrhoup}
\renewcommand{\varsigma}{\varsigmaup}
\renewcommand{\varepsilon}{\varepsilonup}
\renewcommand{\varphi}{\varphiup}